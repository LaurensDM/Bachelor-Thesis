%%=============================================================================
%% Inleiding
%%=============================================================================

\chapter{\IfLanguageName{dutch}{Inleiding}{Introduction}}%
\label{ch:inleiding}

\section{\IfLanguageName{dutch}{Probleemstelling}{Problem Statement}}%
\label{sec:probleemstelling}

% Uit je probleemstelling moet duidelijk zijn dat je onderzoek een meerwaarde heeft voor een concrete doelgroep. De doelgroep moet goed gedefinieerd en afgelijnd zijn. Doelgroepen als ``bedrijven,'' ``KMO's'', systeembeheerders, enz.~zijn nog te vaag. Als je een lijstje kan maken van de personen/organisaties die een meerwaarde zullen vinden in deze bachelorproef (dit is eigenlijk je steekproefkader), dan is dat een indicatie dat de doelgroep goed gedefinieerd is. Dit kan een enkel bedrijf zijn of zelfs één persoon (je co-promotor/opdrachtgever).
In de moderne wereld,
waarin tech\-no\-lo\-gie \@ steeds belangrijker wordt,
is het niet meer dan normaal dat er ook technologie bestaat voor sporters.
Zo zijn er al verschillende apps beschikbaar voor lopers, zoals Strava, Runkeeper en Nike Run Club.
De functies dat deze apps aanbieden zijn echter niet altijd gefocust op het genereren van routes of zijn niet toegankelijk via de gratis versie van de app.
In sommige gevallen moet er een abonnement worden afgesloten om gebruik te kunnen maken van alle functies. Een API, oftewel een interface, maakt communicatie tussen verschillende applicaties mogelijk.
Welke publieke, en gratis, API's zijn er beschikbaar om routes te genereren? Welke combinatie van deze API's is het meest geschikt voor het genereren van routes voor loopfanaten? Welke functies bieden populaire route-apps aan? Welke parameters willen loopfanaten aan een route koppelen? Dit zijn de vragen die in dit onderzoek beantwoord zullen worden.


\section{\IfLanguageName{dutch}{Onderzoeksvraag}{Research question}}%
\label{sec:onderzoeksvraag}

Hoe kunnen online tools gecombineerd worden om een gratis, persoonaliseerbare routeapp te ontwikkelen voor loopfanaten, 
en welke combinatie van deze tools is hiervoor optimaal?

\section{\IfLanguageName{dutch}{Onderzoeksdoelstelling}{Research objective}}%
\label{sec:onderzoeksdoelstelling}

% Wat is het beoogde resultaat van je bachelorproef? Wat zijn de criteria voor succes? Beschrijf die zo concreet mogelijk. Gaat het bv.\ om een proof-of-concept, een prototype, een verslag met aanbevelingen, een vergelijkende studie, enz.
Dit onderzoek concentreert zich op het verzamelen van diverse online tools voor het genereren van een route, in de vorm van gratis open API's, 
met als doel ze te integreren in een kosteloze route-app voor loopfanaten.
Om dit te realiseren zal er een proof of concept ontwikkeld worden die gebruik maakt van de gekozen API's. 
De applicatie zal ontwikkeld worden in React Native en Node.js. React Native is een framework 
dat het mogelijk maakt om native apps te ontwikkelen voor Android en iOS\@. 
Node.js is een JavaScript runtime die het mogelijk maakt om JavaScript code te schrijven buiten de browser.
Deze proof of concept zal beschikbaar zijn op zowel Android- als iOS-systemen. Weinig zaken zijn effectief gratis, 
daarom zal er ook een kostenanalyse uitgevoerd worden om de haalbaarheid van de lancering en het onderhoud van de applicatie te beoordelen.

\section{\IfLanguageName{dutch}{Opzet van deze bachelorproef}{Structure of this bachelor thesis}}%
\label{sec:opzet-bachelorproef}

% Het is gebruikelijk aan het einde van de inleiding een overzicht te
% geven van de opbouw van de rest van de tekst. Deze sectie bevat al een aanzet
% die je kan aanvullen/aanpassen in functie van je eigen tekst.

De rest van deze bachelorproef is als volgt opgebouwd:

In Hoofdstuk~\ref{ch:stand-van-zaken} wordt een overzicht gegeven van de stand van zaken binnen het onderzoeksdomein, op basis van een literatuurstudie.

In Hoofdstuk~\ref{ch:methodologie} wordt de methodologie toegelicht en worden de gebruikte onderzoekstechnieken besproken om een antwoord te kunnen formuleren op de onderzoeksvragen.

% TODO: Vul hier aan voor je eigen hoofstukken, één of twee zinnen per hoofdstuk

In Hoofdstuk~\ref{ch:uitwerking} wordt de proof of concept besproken en worden de resultaten van de ontwikkeling van de applicatie toegelicht.

In Hoofdstuk~\ref{ch:resultaten} worden de resultaten van het genereren van enkele routes besproken.

In Hoofdstuk~\ref{ch:kostenanalyse} wordt een kostenanalyse uitgevoerd voor de ontwikkeling en het onderhoud van de Proof of Concept.

In Hoofdstuk~\ref{ch:verbeteringen} wordt een diepgaande verkenning van potentiële verbeteringen voor de applicatie gegeven, gebaseerd op een grondige analyse van de literatuur en inzichten verworven tijdens de ontwikkelingsfase.

In Hoofdstuk~\ref{ch:conclusie}, tenslotte, wordt de conclusie gegeven en een antwoord geformuleerd op de onderzoeksvragen. Daarbij wordt ook een aanzet gegeven voor toekomstig onderzoek binnen dit domein.