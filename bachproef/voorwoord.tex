%%=============================================================================
%% Voorwoord
%%=============================================================================

\chapter*{\IfLanguageName{dutch}{Woord vooraf}{Preface}}%
\label{ch:voorwoord}

%% TODO:
%% Het voorwoord is het enige deel van de bachelorproef waar je vanuit je
%% eigen standpunt (``ik-vorm'') mag schrijven. Je kan hier bv. motiveren
%% waarom jij het onderwerp wil bespreken.
%% Vergeet ook niet te bedanken wie je geholpen/gesteund/... heeft
Met trots presenteer ik deze bachelorproef, het resultaat van een veel onderzoek en ontwikkeling. 
In deze bachelorproef heb ik geprobeerd om een brug te slaan tussen theoretische kennis en praktische toepassingen.
Het onderwerp van deze thesis, het genereren van looproutes, spreekt mij persoonlijk enorm aan. 
Niet alleen omdat het relevant is voor de sportwereld, maar ook omdat het een breder maatschappelijk belang dient, ik ken persoonlijk ook een aantal lopers die baat zouden hebben aan deze applicatie. 
Ik hoop dat deze bachelorproef een waardevolle bijdrage levert aan het onderzoek naar en de verdere ontwikkeling van looproutes.
Mijn oprechte dank gaat uit naar mijn promotor, Giselle Vercauteren, en co-promotor, Manu De Buck, 
voor hun voortdurende begeleiding, expertise en inspiratie tijdens dit proces. 
Zonder hun begeleiding en ondersteuning zou dit werk niet tot stand zijn gekomen.
Ook wil ik mijn medestudenten bedanken voor de waardevolle discussies 
en gedeelde ervaringen die hebben bijgedragen aan mijn persoonlijke en professionele groei.
Het is mijn hoop dat deze bachelorproef niet alleen academische waarde heeft, 
maar ook praktische inzichten biedt en de interesse wekt van degenen die geïnteresseerd zijn in het onderwerp.
