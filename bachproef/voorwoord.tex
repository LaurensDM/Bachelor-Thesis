%%=============================================================================
%% Voorwoord
%%=============================================================================

\chapter*{\IfLanguageName{dutch}{Woord vooraf}{Preface}}%
\label{ch:voorwoord}

%% TODO:
%% Het voorwoord is het enige deel van de bachelorproef waar je vanuit je
%% eigen standpunt (``ik-vorm'') mag schrijven. Je kan hier bv. motiveren
%% waarom jij het onderwerp wil bespreken.
%% Vergeet ook niet te bedanken wie je geholpen/gesteund/... heeft
Het is met een gevoel van voldoening dat ik deze bachelorproef aan u voorleg, het resultaat van een boeiende en uitdagende periode van onderzoek en ontwikkeling. Als auteur heb ik getracht om een brug te slaan tussen theoretische inzichten en praktische toepassingen.

Het onderwerp van deze bachelorproef, het genereren van looproutes, is een onderwerp dat mij persoonlijk erg interesseert. Het is een onderwerp dat niet alleen relevant is voor de sportwereld, maar ook voor de bredere maatschappij. Het is mijn hoop dat deze bachelorproef een bijdrage levert aan het onderzoek naar en de ontwikkeling van looproutes.

Mijn oprechte dank gaat uit naar mijn promotor, Giselle Vercauteren, en co-promotor, Manu De Buck, voor hun voortdurende begeleiding, deskundigheid en inspiratie gedurende de ontwikkeling van deze bachelorproef. Zonder hun begeleiding en ondersteuning zou dit werk niet tot stand zijn gekomen.

Ik wil ook mijn medestudenten bedanken voor de waardevolle discussies en gedeelde ervaringen die hebben bijgedragen aan mijn persoonlijke en professionele groei.

Het is mijn oprechte hoop dat deze bachelorproef niet alleen academische waarde heeft, maar ook praktische inzichten biedt en de interesse wekt van zij die geïnteresseerd zijn in het onderwerp.

