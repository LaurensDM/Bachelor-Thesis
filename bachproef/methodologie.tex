%%=============================================================================
%% Methodologie
%%=============================================================================

\chapter{\IfLanguageName{dutch}{Methodologie}{Methodology}}%
\label{ch:methodologie}

%% TODO: In dit hoofstuk geef je een korte toelichting over hoe je te werk bent
%% gegaan. Verdeel je onderzoek in grote fasen, en licht in elke fase toe wat
%% de doelstelling was, welke deliverables daar uit gekomen zijn, en welke
%% onderzoeksmethoden je daarbij toegepast hebt. Verantwoord waarom je
%% op deze manier te werk gegaan bent.
%% 
%% Voorbeelden van zulke fasen zijn: literatuurstudie, opstellen van een
%% requirements-analyse, opstellen long-list (bij vergelijkende studie),
%% selectie van geschikte tools (bij vergelijkende studie, "short-list"),
%% opzetten testopstelling/PoC, uitvoeren testen en verzamelen
%% van resultaten, analyse van resultaten, ...
%%
%% !!!!! LET OP !!!!!
%%
%% Het is uitdrukkelijk NIET de bedoeling dat je het grootste deel van de corpus
%% van je bachelorproef in dit hoofstuk verwerkt! Dit hoofdstuk is eerder een
%% kort overzicht van je plan van aanpak.
%%
%% Maak voor elke fase (behalve het literatuuronderzoek) een NIEUW HOOFDSTUK aan
%% en geef het een gepaste titel.


\section{Literatuurstudie en Probleemanalyse}

De eerste fase van het onderzoek, de literatuurstudie en probleemanalyse, omvat een grondige verkenning van de bestaande problemen met betrekking tot het genereren van looproutes. 
Het doel is om een diepgaand begrip te krijgen van de uitdagingen waarmee hardlopers worden geconfronteerd bij het vinden van geschikte routes, 
evenals om inzicht te krijgen in de huidige stand van zaken van route-apps en de functionaliteiten die zij bieden. Dit wordt bereikt door het raadplegen van wetenschappelijke literatuur, 
technische documentatie en het uitvoeren van interviews met zowel hardlopers als experts op het gebied van routeplanning.

Verschillende parameters worden onderzocht die van belang zijn bij het genereren van routes, zoals afstand, hoogteverschil, ondergrond, veiligheid en weersomstandigheden. 
Door deze parameters te analyseren, kunnen de specifieke behoeften en voorkeuren van loopfanaten in kaart worden gebracht. Dit resulteert in een gedetailleerde lijst van gewenste features voor de applicatie,
die zal fungeren als leidraad voor de verdere ontwikkeling ervan.

Verder wordt uitgebreid onderzoek gedaan naar beschikbare API's voor het genereren van routes en het opvragen van weersverwachtingen. 
Hierbij wordt niet alleen gekeken naar de functionaliteiten van deze API's, maar ook naar hun technische integratiemogelijkheden in de applicatie. 
Op basis van deze analyse wordt een weloverwogen keuze gemaakt voor de API's die zullen worden gebruikt in het Proof of Concept van de applicatie.

De reden voor deze aanpak is om een stevige basis te leggen voor de ontwikkeling van de applicatie. 
Door een grondige analyse uit te voeren van de problemen en behoeften van de doelgroep, evenals van de beschikbare technologische oplossingen, 
kan de applicatie worden ontworpen en ontwikkeld met de juiste functionaliteiten en technologieën die aansluiten bij de verwachtingen van de gebruiker.

\section{Ontwerp}

Na de literatuurstudie en probleemanalyse volgt de fase van ontwerp, waarbij de architectuur en functionaliteiten van de applicatie worden bepaald. 
Het doel van deze fase is om een duidelijk beeld te krijgen van hoe de applicatie eruit zal zien en welke functionaliteiten deze zal bevatten. 
Dit wordt bereikt door het maken van wireframes en mock-ups van de gebruikersinterface en het definiëren van de interacties en workflows van de applicatie.

Het ontwerp van de applicatie is gebaseerd op de gewenste features die zijn geïdentificeerd in de literatuurstudie en probleemanalyse. 
Hierbij wordt rekening gehouden met de behoeften en voorkeuren van de doelgroep, evenals met de technische mogelijkheden van de geselecteerde API's. 
Het doel is om een intuïtieve en gebruiksvriendelijke applicatie te ontwerpen die voldoet aan de verwachtingen van de gebruiker en aansluit bij de huidige trends in de markt voor routeplanning-apps.

Het ontwerp van de applicatie zal iteratief zijn, waarbij regelmatig feedback wordt verzameld van gebruikers en stakeholders. 
Op basis van deze feedback kunnen aanpassingen worden gemaakt aan het ontwerp om ervoor te zorgen dat de applicatie optimaal aansluit bij de behoeften van de gebruiker. 
Dit kan onder meer het aanpassen van de lay-out, het toevoegen van nieuwe features of het verbeteren van de gebruikerservaring omvatten.

Door een grondige ontwerpbenadering te hanteren, kan worden gegarandeerd dat de uiteindelijke applicatie goed doordacht is en naadloos aansluit bij de behoeften van de gebruiker. 
Dit zal bijdragen aan een positieve gebruikerservaring en het succes van de applicatie op de markt voor routeplanning-apps.

\section{Ontwikkeling en Implementatie}

Na het ontwerp volgt de fase van ontwikkeling en implementatie. Hier wordt een Proof of Concept ontwikkeld, bestaande uit zowel een online backend als een mobiele applicatie. 
De backend integreert de geselecteerde API's en is verantwoordelijk voor het genereren van routes en weersverwachtingen. 
De mobiele applicatie vormt de gebruikersinterface van de applicatie en stelt gebruikers in staat om routes te plannen en weerinformatie te bekijken.

De ontwikkeling van deze Proof of Concept vereist een iteratieve benadering, waarbij regelmatig feedback wordt verzameld van gebruikers en stakeholders. 
Op basis van deze feedback kunnen aanpassingen worden gemaakt aan het ontwerp en de functionaliteiten van de applicatie. 
Het doel is om een werkende applicatie te ontwikkelen die voldoet aan de gewenste features en gebruiksvriendelijk is voor de eindgebruiker.

De keuze voor het ontwikkelen van een Proof of Concept is om de levensvatbaarheid van de applicatie te valideren voordat er grote investeringen worden gedaan in verdere ontwikkeling. 
Door een Proof of Concept te ontwikkelen, kunnen eventuele technische en gebruikersgerelateerde uitdagingen in een vroeg stadium worden geïdentificeerd en aangepakt.

\section{Kostenanalyse}

Na de ontwikkeling van de Proof of Concept wordt een kostenanalyse uitgevoerd om de haalbaarheid van de lancering en het onderhoud van de applicatie te beoordelen. 
Hierbij wordt specifiek onderzoek gedaan naar de kosten van de geselecteerde API's, het hosten van de backend en het publiceren van de applicatie in de App Store en Google Play Store.

De kostenanalyse is essentieel om inzicht te krijgen in de financiële aspecten van het lanceren en onderhouden van de applicatie. 
Op basis van deze analyse kan worden bepaald of de applicatie economisch haalbaar is en welke financierings- en inkomstenstromen moeten worden overwogen.

\section{Evalueren}

De applicatie wordt geëvalueerd om te bepalen of deze voldoet aan de verwachtingen van de gebruiker. 
Hierbij wordt gekeken naar de gebruiksvriendelijkheid, de functionaliteiten en de prestaties van de applicatie. 
Deze evaluatie resulteert in een conclusie over de haalbaarheid van de applicatie en eventuele aanbevelingen voor verdere verbetering.

De evaluatie is cruciaal om inzicht te krijgen in de sterke punten en zwakke punten van de applicatie, evenals om te bepalen of deze klaar is voor lancering. 
Op basis van de evaluatie kunnen beslissingen worden genomen over eventuele aanpassingen en verbeteringen die moeten worden doorgevoerd voordat de applicatie wordt gelanceerd.

\section{Conclusie en Toekomstig Onderzoek}

Tot slot wordt gekeken naar toekomstig onderzoek, waarbij mogelijke uitbreidingen van de applicatie en verbeteringen worden onderzocht. 
Hierbij wordt ook gekeken naar mogelijke onderzoeken die kunnen worden uitgevoerd om de applicatie verder te verbeteren en te optimaliseren naar de behoeften van de gebruiker.

Het identificeren van mogelijke toekomstige onderzoeksthema's is essentieel om de ontwikkeling van de applicatie voort te zetten 
en ervoor te zorgen dat deze blijft evolueren om aan de veranderende behoeften van de gebruikers te voldoen. Dit omvat onder meer het verkennen van nieuwe technologieën, 
het uitvoeren van gebruikerstests en het monitoren van trends in de markt voor routeplanning-apps.

