%%=============================================================================
%% Methodologie
%%=============================================================================

\chapter{\IfLanguageName{dutch}{Methodologie}{Methodology}}%
\label{ch:methodologie}

%% TODO: In dit hoofstuk geef je een korte toelichting over hoe je te werk bent
%% gegaan. Verdeel je onderzoek in grote fasen, en licht in elke fase toe wat
%% de doelstelling was, welke deliverables daar uit gekomen zijn, en welke
%% onderzoeksmethoden je daarbij toegepast hebt. Verantwoord waarom je
%% op deze manier te werk gegaan bent.
%% 
%% Voorbeelden van zulke fasen zijn: literatuurstudie, opstellen van een
%% requirements-analyse, opstellen long-list (bij vergelijkende studie),
%% selectie van geschikte tools (bij vergelijkende studie, "short-list"),
%% opzetten testopstelling/PoC, uitvoeren testen en verzamelen
%% van resultaten, analyse van resultaten, ...
%%
%% !!!!! LET OP !!!!!
%%
%% Het is uitdrukkelijk NIET de bedoeling dat je het grootste deel van de corpus
%% van je bachelorproef in dit hoofstuk verwerkt! Dit hoofdstuk is eerder een
%% kort overzicht van je plan van aanpak.
%%
%% Maak voor elke fase (behalve het literatuuronderzoek) een NIEUW HOOFDSTUK aan
%% en geef het een gepaste titel.

- Analyse van parameters voor route-generatie
- Analyse van API's
- Analyse van bestaande route-apps
- Proof of Concept
- Improvements

De aanpak van dit onderzoek bestaat uit drie delen: een analyse van API's voor het genereren van routes, een analyse van bestaande route-apps, en een \emph{Proof of Concept} voor het genereren van routes.

\section{Analyse van API's}

Eerst wordt een grondige analyse uitgevoerd van beschikbare API's voor het genereren van routes.
Daarna wordt een vergelijking gemaakt tussen verschillende API's,
waarbij keuzes worden gebaseerd op factoren zoals features, gebruikslimieten,
documentatieduidelijkheid, en compatibiliteit met een Node.js server.
Het genereren van gewenste routes vereist een combinatie van verschillende API's,
wat nader onderzocht zal worden. Hiernaast zijn er ook API's nodig voor het opvragen van weersverwachtingen.
Afhankelijk van de benodigde features, kunnen één of meerdere API's worden gebruikt.
De analyse van API's en hun features is al gestart in de state-of-the-art sectie,
onder het hoofdstuk '\emph{Bestaand onderzoek}'.
Concreet zal een 'hoofd-API' worden gekozen als basis voor route-generatie,
aangevuld met andere API's om gewenste features te bereiken.
Een ideale hoofd-API voor een looproute-app zou moeten beschikken
over functies voor het genereren van routes op basis van diverse parameters, flexibiliteit in kaartgegevens,
gedetailleerde documentatie, schaalbaarheid, duidelijke gebruikslimieten en tarieven, snelle responsiviteit,
aanpasbaarheid, en vooruitziende mogelijkheden voor toekomstige uitbreidingen.
De combinatie van de OpenStreetMap API en de Google Maps API is reeds succesvol gebleken voor route-generatie \autocite{Loepp2018},
en zal daarom grondig worden onderzocht.

\section{Analyse van bestaande route-apps}

Er wordt ook een analyse gemaakt van bestaande route-apps.
Welke functies bieden ze aan? Welke parameters willen loopfanaten aan een route koppelen? Hoe ziet de UI/UX eruit?
Deze analyse zal worden uitgevoerd door het bestuderen van de verschillende route-apps 
en eventueel het uitvoeren van een enquête met gebruikers van loop-apps.
Dit zal de basis vormen voor het bepalen van gewenste features en het ontwerp van de applicatie. 

\section{Proof of Concept}

Na deze uitgebreide analyse-fasen wordt een \emph{Proof of Concept} ontwikkeld.
Voor het \emph{Proof of Concept} wordt een applicatie ontwikkeld met gebruik van de gekozen API's.
Het Proof of Concept bestaat uit twee delen: een online backend ontwikkeld in Node.js
en een mobiele applicatie ontwikkeld in React Native. 
De backend zal de verschillende API's integreren en is verantwoordelijk voor 
de generatie van routes en het opvragen van weersverwachtingen.
De mobiele applicatie zal de gebruikersinterface vormen voor de applicatie.
Als een laatste stap wordt een kostenanalyse uitgevoerd om de haalbaarheid van de lancering en het onderhoud van de applicatie te beoordelen.
Hierbij wordt concreet onderzoek gedaan naar de kosten van de verschillende API's voor een bepaald aantal requests, 
de kosten van het hosten van de backend, en de kosten van het publiceren van de applicatie in de App Store en Google Play Store.
Verder worden er mogelijkheden onderzocht om deze kosten te dekken, zoals het toevoegen van advertenties.
De applicatie zal beschikbaar zijn op zowel Android- als iOS-systemen.

