%%=============================================================================
%% Conclusie
%%=============================================================================

\chapter{Conclusie}%
\label{ch:conclusie}

% TODO: Trek een duidelijke conclusie, in de vorm van een antwoord op de
% onderzoeksvra(a)g(en). Wat was jouw bijdrage aan het onderzoeksdomein en
% hoe biedt dit meerwaarde aan het vakgebied/doelgroep? 
% Reflecteer kritisch over het resultaat. In Engelse teksten wordt deze sectie
% ``Discussion'' genoemd. Had je deze uitkomst verwacht? Zijn er zaken die nog
% niet duidelijk zijn?
% Heeft het onderzoek geleid tot nieuwe vragen die uitnodigen tot verder 
%onderzoek?

De ontwikkeling van de Proof of Concept heeft aangetoond dat het mogelijk is om een eenvoudige maar functionele route-applicatie te bouwen, die gebruikers in staat stelt om routes te plannen en op te slaan met behulp van diverse online tools, zoals Google Maps, OpenStreetMap en OpenWeatherMap.

\vspace{1cm}


De keuze om de applicatie te ontwikkelen voor het Android-platform biedt een breed bereik, maar er is ook potentieel voor compatibiliteit met iOS-apparaten, hoewel deze mogelijkheid niet uitgebreid is getest. Het gebruik van React Native als ontwikkelingsframework maakt het mogelijk om een native-achtige ervaring te bieden aan gebruikers van verschillende mobiele apparaten.

\vspace{1cm}


De integratie van een Express.js backend zorgt voor de benodigde connectiviteit met externe tools en services, waardoor de applicatie haar functionaliteit kan uitbreiden en verbeteren. Het testen van de applicatie op zowel fysieke apparaten als emulators draagt bij aan de betrouwbaarheid en prestaties van de applicatie op verschillende platforms.

\vspace{1cm}


Een belangrijk aspect van de Proof of Concept is de mogelijkheid om de applicatie gratis aan te bieden aan gebruikers, hoewel er mogelijkerwijs extra kosten verbonden zijn aan het toenemende aantal gebruikers. Het bieden van een gratis versie kan de gebruikersacquisitie bevorderen en de adoptie van de applicatie stimuleren.

\vspace{1cm}


In de toekomst kunnen verdere verbeteringen en uitbreidingen worden overwogen, zoals het toevoegen van meer geavanceerde functionaliteiten, verbeterde ondersteuning voor meerdere platforms, en meer integraties met andere platforms die een grote rol spelen in de wereld van loopfanaten. Met voortdurende iteratie en feedback van gebruikers kan de Proof of Concept evolueren tot een volwaardige en veelgebruikte route-applicatie in de markt van mobiele navigatietools.


