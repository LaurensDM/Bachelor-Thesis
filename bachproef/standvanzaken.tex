\chapter{\IfLanguageName{dutch}{Stand van zaken}{State of the art}}%
\label{ch:stand-van-zaken}

% Tip: Begin elk hoofdstuk met een paragraaf inleiding die beschrijft hoe
% dit hoofdstuk past binnen het geheel van de bachelorproef. Geef in het
% bijzonder aan wat de link is met het vorige en volgende hoofdstuk.

% Pas na deze inleidende paragraaf komt de eerste sectiehoofding.

De voortdurende vooruitgang in digitale technologieën heeft geleid tot een overvloed aan online hulpmiddelen en toepassingen,
waaronder diverse apps voor het begeleiden van fysieke activiteiten.
Deze sectie over de huidige stand van zaken onderzoekt de integratie van online hulpmiddelen voor de ontwikkeling van kosteloze,
op maat gemaakte route-applicaties.

\subsection{Populaire apps voor lopers}
Er bestaan reeds verschillende applicaties voor lopers. Enkele populaire apps zijn:
Runna, Strava, Nike Run Club, Map My Run by Under Armour, Runkeeper, Peloton, Stride, Apple Fitness Plus en anderen \autocite{Downey2023}.
Uit deze lijst van zogezegde "beste"\@ apps voor lopers,
is er geen enkele app die volledig gratis is \textbf{en} een route kan genereren op basis van verschillende parameters.
Echter hebben deze apps wel andere interessante functies, zoals het bijhouden van statistieken, het delen van routes met andere gebruikers,
het aanmaken van groepen, het volgen van andere gebruikers, het aanmaken van een trainingsschema \ldots \@
Deze functies kunnen ook interessant zijn om te integreren in de applicatie.

\subsection{API's voor het genereren van routes}
Er zijn verschillende API's beschikbaar voor het genereren van routes.
Enkele voorbeelden zijn: Google Maps, Geoapify, Mapbox, OpenRouteService, Here, GraphHopper, OpenStreetMap \ldots \@
Deze API's zijn allemaal gratis te gebruiken, maar hebben een limiet voor het aantal requests per dag of per maand,
of bieden minder functies in de gratis versie. De limiet verschilt per API, maar is meestal voldoende voor een applicatie die nog in ontwikkeling is.
Elke API heeft zijn eigen voor- en nadelen. Zo is Google Maps een zeer populaire API,
maar is het niet mogelijk om een route te genereren op basis van het aantal hoogtemeters.

\subsection{API's voor het opvragen van weersverwachtingen}
Er zijn verschillende API's beschikbaar voor het opvragen van weersverwachtingen.
Enkele voorbeelden zijn: OpenWeather, Weather API, Weatherbit, Weatherstack \ldots \@
Zoals bij de API's voor het genereren van routes, zijn deze API's ook gratis te gebruiken,
maar hebben ze wel een limiet op het aantal requests per dag of per maand, of hebben ze minder features voor een gratis versie.
Voor een simpele weersverwachting is \emph{één} van deze API's voldoende, Weather API bevat de meeste features,
dus deze zou zeker geschikt zijn. Voor meer features zoals zonsondergang, zonsopgang, luchtvochtigheid, windrichting \ldots \@
is het nodig om meerdere API's te combineren.

\subsection{Bestaand onderzoek}
Er is al onderzoek gedaan naar het genereren van routes voor lopers.
\textcite{Loepp2018} hebben onderzoek gedaan naar het genereren van routes voor lopers op basis van verschillende parameters.
Hier werd een app, genaamd \emph{Runnerful}, voorgesteld.
Deze maakt gebruik van gemakkelijk toegankelijke kaartgegevensbronnen om routes te genereren met een door de gebruiker gespecificeerde lengte.
Vervolgens werden  deze kandidaat-routes gerangschikt op basis van individuele vereisten door verdere verwerking van de kaartgegevens.
Door middel van kritiek kan de hardloper de resultaten interactief verfijnen.
De API's die gebruikt werden voor het genereren van routes zijn voornamelijk OpenStreetMap API,
met occasionele requests naar de Google Maps API \autocite{Loepp2018}.

% Dit hoofdstuk bevat je literatuurstudie. De inhoud gaat verder op de inleiding, maar zal het onderwerp van de bachelorproef *diepgaand* uitspitten. De bedoeling is dat de lezer na lezing van dit hoofdstuk helemaal op de hoogte is van de huidige stand van zaken (state-of-the-art) in het onderzoeksdomein. Iemand die niet vertrouwd is met het onderwerp, weet nu voldoende om de rest van het verhaal te kunnen volgen, zonder dat die er nog andere informatie moet over opzoeken \autocite{Pollefliet2011}.

% Je verwijst bij elke bewering die je doet, vakterm die je introduceert, enz.\ naar je bronnen. In \LaTeX{} kan dat met het commando \texttt{$\backslash${textcite\{\}}} of \texttt{$\backslash${autocite\{\}}}. Als argument van het commando geef je de ``sleutel'' van een ``record'' in een bibliografische databank in het Bib\LaTeX{}-formaat (een tekstbestand). Als je expliciet naar de auteur verwijst in de zin (narratieve referentie), gebruik je \texttt{$\backslash${}textcite\{\}}. Soms is de auteursnaam niet expliciet een onderdeel van de zin, dan gebruik je \texttt{$\backslash${}autocite\{\}} (referentie tussen haakjes). Dit gebruik je bv.~bij een citaat, of om in het bijschrift van een overgenomen afbeelding, broncode, tabel, enz. te verwijzen naar de bron. In de volgende paragraaf een voorbeeld van elk.

%  schreef een van de standaardwerken over sorteer- en zoekalgoritmen. Experten zijn het erover eens dat cloud computing een interessante opportuniteit vormen, zowel voor gebruikers als voor dienstverleners op vlak van informatietechnologie~\autocite{Creeger2009}.

% Let er ook op: het \texttt{cite}-commando voor de punt, dus binnen de zin. Je verwijst meteen naar een bron in de eerste zin die erop gebaseerd is, dus niet pas op het einde van een paragraaf.

\lipsum[7-20]
