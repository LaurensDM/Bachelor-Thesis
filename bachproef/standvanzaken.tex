\chapter{\IfLanguageName{dutch}{Stand van zaken}{State of the art}}%
\label{ch:stand-van-zaken}

% Tip: Begin elk hoofdstuk met een paragraaf inleiding die beschrijft hoe
% dit hoofdstuk past binnen het geheel van de bachelorproef. Geef in het
% bijzonder aan wat de link is met het vorige en volgende hoofdstuk.

% Pas na deze inleidende paragraaf komt de eerste sectiehoofding.

De voortdurende vooruitgang in digitale technologieën heeft geleid tot een overvloed aan online hulpmiddelen en toepassingen,
waaronder diverse apps voor het begeleiden van fysieke activiteiten.
Deze sectie, over de huidige stand van zaken, onderzoekt de integratie van online tools voor de ontwikkeling van route-applicaties.

    \section{Route generatie}

    Er is al onderzoek gedaan naar het genereren van routes voor lopers. In deze sectie worden enkele van deze onderzoeken besproken.
    
    \textcite{Loepp2018} hebben onderzoek gedaan naar het genereren van routes voor lopers op basis van verschillende parameters.
    Dit zijn de voorgestelde parameters: Lengte, uniekheid, vorm, verlichting, hoogte, vriendelijkheid, voor voetgangers, bochten, natuur en geschiedenis.
    Zo werd een app, genaamd \emph{Runnerful} een prototypische Android-applicatie, voorgesteld. Er wordt gebruikgemaakt van de OpenStreetMap API om kaartgegevens te verzamelen. 
    Met behulp van randannotaties in deze gegevens (bijvoorbeeld om snelwegen te negeren), 
    wordt vervolgens een grafiek gecreëerd om het route-generatiealgoritme toe te passen.
    Om de kortste routes te vinden tussen via-vertices, wordt een gemodificeerd A* zoekalgoritme gebruikt dat knooppunten 
    die al zijn bezocht bestraft (afstanden worden gevarieerd en verschillende startpunten worden ingesteld). 
    Scores worden berekend voor alle criteria: Voor sommige criteria, zoals Lengte of Vorm, worden scores berekend op basis van de grafiekgegevens zelf. 
    Voor andere, zoals Verlichting of Vriendelijkheid voor voetgangers, wordt vertrouwd op randannotaties verstrekt door OpenStreetMap. 
    Voor Hoogte worden bovendien verzoeken naar de Google Maps Hoogte API gestuurd. Wat betreft de hoeveelheid Natuur, wordt rekening gehouden met omliggende gebieden en hun OpenStreetMap-annotaties:
    Met behulp van een straal-casting algoritme wordt bepaald of segmenten bossen, landbouwgrond of stranden doorkruisen. De verhouding van randen waarvoor dit geldt, 
    bepaalt vervolgens de respectieve score. Bovendien wordt de afstand van elk routepunt tot gebieden die water voorstellen berekend. 
    Als gebruikersinvoer vereist de app in eerste instantie alleen de gewenste routelengte. 
    Vervolgens worden aanbevelingen gegenereerd op basis van de huidige GPS-positie. 
     \autocite{Loepp2018}.

    \textcite{Schulze2016} hebben een model voorgesteld voor het rangschikken van routes om de beste en mooiste routes te vinden
    die aan bepaalde basisvereisten voldoen. Hiervoor werd een Android applicatie ontwikkeld die gebruik maakt van OpenStreetMap data.
    De applicatie kan geoptimaliseerde routes vinden met een aanpasbare lengte op elke locatie voor buitenactiviteiten 
    zoals hardlopen, wandelen of een wandeling maken \autocite{Schulze2016}.

    \textcite{Adwinda2020} hebben een Android-applicatie ontwikkeld die feedback geeft over de activiteiten van gebruikers tijdens het hardlopen.
    Dit zijn de front-end features:
    \begin{itemize}
        \item Locatietracking
        \item Berekening van de hartslag
        \item Activiteitengrafiek
        \item Vrienden
        \item Geplande hardlooproutes
        \item Hardlopen met vrienden
        \item Pushmeldingen
        \item Gamification
        \item Delen op Facebook
        \item Nieuws en berichten
    \end{itemize}
    Dit zijn de voordelen dat een gebruiker kan krijgen van de applicatie:
    \begin{itemize}
        \item Gebruikers kunnen informatie krijgen over de duur van hun hardloopactiviteit, tempo, afstand, verbrande calorieën en hun hardlooproutes
        \item Gebruikers kunnen hun hartslag zien
        \item Gebruikers kunnen hun hardloopgeschiedenis delen op Facebook
        \item Gebruikers kunnen communiceren met andere hardlopers via sociale functies (bijv. vrienden en feeds)
        \item Gebruikers kunnen de realtime locatie en het tempo van hun vrienden volgen
        \item Gebruikers kunnen hun tempo geschiedenissen zien in de vorm van een grafiek
        \item Gebruikers kunnen hun prestaties zien
        \item Gebruikers kunnen nieuws lezen dat beschikbaar is in de applicatie
    \end{itemize}
    De applicatie werd ontwikkeld in Android en maakt gebruik van de Google Maps API, verder werd er gebruik gemaakt van Firebase voor de database
    en messaging functies.
    \autocite{Adwinda2020}.

    \textcite{Gallo2020} hebben een navigatie systeem ontwikkeld dat head scanning gebruikt voor navigatie feedback.

    \textcite{Novack2018} hebben een systeem ontwikkeld dat gebruik maakt van een algoritme dat rekening houdt met de voorkeuren van de gebruiker.
    De voorkeuren zijn, onder andere, het vermijden van drukke straten of het kiezen van een route met een mooi uitzicht.
    Het systeem maakt gebruik van OpenStreetMap data \autocite{Novack2018}.

    \section{Doelpubliek}

    Welk doelpubliek heeft interesse in een route-genererende applicatie? Welke features zijn belangrijk voor dit doelpubliek?
    Hier wordt onderzocht welke doelgroepen interesse hebben in een dergelijke applicatie.

    De heterogeniteit onder hardlopers maakt het nuttig en aantrekkelijk om hen in groepen te segmenteren om hun AOI's te begrijpen. 
    Segmentatie van consumenten in sport is uitgebreid gedocumenteerd, waarbij studies meestal onderscheid maken tussen consumenten 
    op basis van demografische factoren. Naast demografische kenmerken blijken gedrags- en psychografische variabelen ook 
    verschillende soorten marathonlopers te onderscheiden. Verschillende studies hebben psychografische variabelen, zoals AIO's, 
    gebruikt om hardlopers te clusteren. Verschillende aspecten zoals gezondheid, hardloopidentiteit, persoonlijke doelen, sociale aspecten, 
    verslaving aan hardlopen, commitment, competitie en gemak van de praktijk worden gebruikt om hardlopers te segmenteren.
    Deze studies benadrukken allemaal het belang van AIO's om waardevolle inzichten te krijgen in de behoeften van hardlopers 
    en om typologieën van hardlopers te creëren \textcite{Janssen2020}. 

    In dit onderzoek werd een online vragenlijst, de Eindhoven Running Survey 2016 (ERS2016), 
    gebruikt om gegevens te verzamelen onder deelnemers van het hardloopevenement Eindhoven Marathon. 
    Een totaal van 3727 deelnemers voltooide de vragenlijst volledig (responspercentage van 20,4 \%). 
    Deelnemers waren verdeeld over verschillende afstanden, met de gemiddelde leeftijd van 42,2 jaar. 
    Ongeveer een derde van de deelnemers was vrouw, bijna negen van de tien waren werkzaam en de meerderheid had hoger onderwijs genoten. 
    De sociodemografische kenmerken van de respondenten waren vergelijkbaar met eerdere grootschalige hardloopstudies in West-Europa.\textcite{Janssen2020}.
    Dit betekent dat de resultaten van dit onderzoek waarschijnlijk representatief zijn voor de hardlooppopulatie in West-Europa.

    In dit onderzoek werden 4 types hardlopers geïdentificeerd:
    \begin{itemize}
        \item Recreatieve individuele hardlopers
        \item Sociale competitieve hardlopers
        \item Individuele competitieve hardlopers
        \item Toegewijde hardlopers
    \end{itemize}

    Binnen deze 4 groepen van hardlopers zijn er verschillen in het gebruik van apps en andere apparatuur. 
    De resultaten toonden significante variaties (alleen significante effecten worden beschreven). 
    In relatieve termen waren recreatieve individuele hardlopers de meest enthousiaste app-gebruikers
    en de kleinste groep gebruikers van sporthorloges, terwijl sociale competitieve hardlopers minder app-gebruikers 
    omvatten dan recreatieve individuen, ongeveer hetzelfde als individuele competitieve
    en meer dan toegewijde hardlopers. Sociale competitieve hardlopers omvatten meer gebruikers van sporthorloges 
    dan de recreatieve individuele groep, en minder dan zowel individuele competitieve
    als toegewijde groepen. De laagste proportie niet-gebruikers werd gevonden 
    onder de individuele competitieve hardlopers in vergelijking met de andere types, 
    terwijl zij en de toegewijde hardlopers de hoogste acceptatie van sporthorloges hadden. 
    Ten slotte hadden de toegewijde hardlopers de minste app-gebruikers, 
    en samen met de individuele competitieve hardlopers hadden zij de hoogste proportie gebruikers van sporthorloges.

    De resultaten van dit onderzoek tonen aan dat er verschillende groepen hardlopers zijn die verschillende behoeften hebben. 
    Het is belangrijk om deze verschillen in overweging te nemen bij het ontwikkelen van een route-genererende applicatie.
    Zo kan er bijvoorbeeld een functie toegevoegd worden die het mogelijk maakt om routes te delen met vrienden,
    of een functie die het mogelijk maakt om een trainingsschema aan te maken.

    \section{Design overwegingen}

    In de sectie over Doelpubliek werd reeds besproken dat er verschillende groepen hardlopers zijn die verschillende behoeften hebben.
    Hieronder zitten niet per se enkel verschillen in het gebruik van apps en andere apparatuur, maar ook verschillen in de manier waarop ze gemotiveerd worden.
    Het is belangrijk om deze verschillen in overweging te nemen bij het ontwikkelen van een route-genererende applicatie.

    Hardlopen wordt gekenmerkt door zijn lage drempel en aantrekkelijkheid voor een breed scala aan mensen. 
    Deze heterogeniteit onder hardlopers komt ook tot uiting in de grote aantallen traditionele en thematische hardloopevenementen. 
    Er is echter een hoog uitvalpercentage onder amateurhardlopers, voornamelijk als gevolg van hardloopgerelateerde blessures en verlies van motivatie. 
    Een belangrijke uitdaging blijft het omzetten van intenties in daadwerkelijk langdurig hardloopgedrag. 
    Barrières en drijfveren tussen het maken van de intentie om te gaan hardlopen en het daadwerkelijke hardlopen zelf spelen hierbij een rol. 
    Begrip van hoe amateurhardlopers dagelijkse barrières ervaren en hoe dit hun potentiële runs beïnvloedt, kan waardevol zijn 
    om te identificeren hoe ontwerp- en interactieve technologieën hardlopers beter kunnen ondersteunen voorbij de eigenlijke run.\textcite{Menheere2020}.

    In het onderzoek van \textcite{Menheere2020} werden verschillende ontwerpaanbevelingen gedaan. Hieronder worden enkele van deze aanbevelingen besproken.

    Ontwerpaanbeveling 1: Begeleid zelfspraak en versterk de verwachte beloning van het hardlopen door middel van ontwerp. 
    Illustratief ontwerpconcept: Een interactieve sportmaatje dat dit gesprek begint kan helpen om deze zelfspraak naar een daadwerkelijk gesprek te verplaatsen, 
    anticipatiegevoelens stimuleren en zo bijdragen aan het overtreffen van drijfveren ten opzichte van barrières. 
    Om te meten of het maatje wordt vastgehouden, kan de Hexiwear-prototypingtool worden gebruikt, met een geïntegreerde versnellingsmeter, 
    trilmotor, Bluetooth low energy en een ingebouwde batterij. Een andere strategie zou kunnen bestaan uit het verminderen van de hoeveelheid 
    negatieve zelfspraak door middel van een interactief apparaat dat zelfbewustzijn ondersteunt en die negatieve gedachten omkeert.

    Ontwerpaanbeveling 2: Maak de voorbereidingsrituelen interessanter of plezieriger door middel van ontwerp. 
    Illustratief ontwerpconcept: Een interactieve kledinghanger die de gebruiker overtuigt om hun sportkleding aan te trekken. 
    De gebruiker hangt hun sportkleding aan de hanger, die hardloopintenties detecteert via een verbinding met de agenda van de gebruikers. 
    Gedurende de dag zal de hanger langzaam beginnen te krimpen door de armen van de hanger aan servomotoren te verbinden. 
    Als de gebruiker de kleding uittrekt om te gaan sporten, verschijnt er een motiverend citaat op een geïntegreerd e-inktscherm. 
    Echter, wanneer de sportkleding niet op tijd wordt uitgetrokken, zal de grootte van de hanger een punt bereiken waarop de sportkleding op de grond valt.

    Ontwerpaanbeveling 3: Bied tools aan om hardlopers te helpen hun hardloopsessie van tevoren te visualiseren. 
    Illustratief ontwerpconcept: Een multisensorisch object dat sensaties oproept die verband houden met iemands persoonlijke hardloopervaring. 
    Tussen de runs door zal het object het geluid van je voetstappen en omgeving afspelen, afhankelijk van de vorige hardlooproute. 
    Het object zal ook natuurlijke geuren verspreiden (bijvoorbeeld bomen, modder, gras) en lichtpatronen gebruiken om hardloopbeelden op te roepen.

    Ontwerpaanbeveling 4: Help hardlopers negatieve emoties, zoals frustratie en onzekerheid, die ze ervaren tijdens het hardlopen te overwinnen 
    door middel van ontwerp. 
    Illustratief ontwerpconcept: Een bestaand concept op de markt dat deze ontwerpuitdaging aanpakt, is 'Zombies Run!', 
    een gegamificeerde applicatie die de hardloper onderdompelt in een post-apocalyptische omgeving. 
    Door hun missie en muziek via koptelefoon te horen, moet de speler zombies vermijden en goederen verzamelen om te overleven. 
    Het gamificeren van de hardloopinspanning transformeert negatieve emoties in positieve gevoelens \autocite{Menheere2020}.

    \section{Technologie}

    Verschillende technologieën kunnnen gebruikt worden voor het ontwikkelen van een route-genererende applicatie. 
    Hieronder worden enkele van deze technologieën besproken en geëvalueerd.

    - React Native
    - Flutter
    - OpenStreetMap API
    - Google Maps API
    - Nodejs

    Voor het ontwikkelen van een route-genererende applicatie kunnen verschillende technologieën gebruikt worden. 
    Om een applicatie te ontwikkelen die zowel op Android als op iOS werkt, kan er gebruik gemaakt worden van React Native of Flutter.
    Deze technologieën maken het mogelijk om een applicatie te ontwikkelen die op beide platformen werkt.
    Een andere optie is om de applicatie te ontwikkelen in Java of Kotlin voor Android en in Swift voor iOS,
    maar dit zou betekenen dat de applicatie twee keer ontwikkeld moet worden.

    Flutter is een open-source UI toolkit van Google voor het bouwen van natively gecompileerde applicaties voor mobiele, web- en desktop vanuit één codebase.

    React Native is een open-source framework voor het ontwikkelen van mobiele applicaties.
    Het werd ontwikkeld door Facebook en is gebaseerd op JavaScript.

    OpenStreetMap is een open-source kaartendatabase die door iedereen kan worden bewerkt.
    Het is een alternatief voor Google Maps en kan gebruikt worden om kaarten te genereren.

    Google Maps API is een set van API's die ontwikkeld zijn door Google om kaarten te integreren in applicaties.
    Deze API's kunnen gebruikt worden om kaarten te genereren, routes te berekenen, locaties te zoeken, enz.

    Nodejs is een open-source, cross-platform JavaScript runtime om applicaties te bouwen. Het kan gebruikt worden om een server te bouwen voor de applicatie.
    Zowel React Native als Flutter kunnen gebruikt worden in combinatie met Nodejs om een volledige applicatie te ontwikkelen.
    OpenStreetMap API en Google Maps API bieden beiden npm packages aan die gebruikt kunnen worden in combinatie met Nodejs.


\section{Populaire apps voor lopers}

Er bestaan reeds verschillende applicaties voor lopers. Enkele populaire apps zijn:
Runna, Strava, Nike Run Club, Map My Run by Under Armour, Runkeeper, Peloton, Stride, Apple Fitness Plus en anderen \autocite{Downey2023}.
Uit deze lijst van zogezegde "beste"\@ apps voor lopers,
is er geen enkele app die volledig gratis is \textbf{en} een route kan genereren op basis van verschillende parameters.
Echter hebben deze apps wel andere interessante functies, zoals het bijhouden van statistieken, het delen van routes met andere gebruikers,
het aanmaken van groepen, het volgen van andere gebruikers, het aanmaken van een trainingsschema \ldots \@
Deze functies kunnen ook interessant zijn om te integreren in de applicatie.



% Dit hoofdstuk bevat je literatuurstudie. De inhoud gaat verder op de inleiding, maar zal het onderwerp van de bachelorproef *diepgaand* uitspitten. De bedoeling is dat de lezer na lezing van dit hoofdstuk helemaal op de hoogte is van de huidige stand van zaken (state-of-the-art) in het onderzoeksdomein. Iemand die niet vertrouwd is met het onderwerp, weet nu voldoende om de rest van het verhaal te kunnen volgen, zonder dat die er nog andere informatie moet over opzoeken \autocite{Pollefliet2011}.

% Je verwijst bij elke bewering die je doet, vakterm die je introduceert, enz.\ naar je bronnen. In \LaTeX{} kan dat met het commando \texttt{$\backslash${textcite\{\}}} of \texttt{$\backslash${autocite\{\}}}. Als argument van het commando geef je de ``sleutel'' van een ``record'' in een bibliografische databank in het Bib\LaTeX{}-formaat (een tekstbestand). Als je expliciet naar de auteur verwijst in de zin (narratieve referentie), gebruik je \texttt{$\backslash${}textcite\{\}}. Soms is de auteursnaam niet expliciet een onderdeel van de zin, dan gebruik je \texttt{$\backslash${}autocite\{\}} (referentie tussen haakjes). Dit gebruik je bv.~bij een citaat, of om in het bijschrift van een overgenomen afbeelding, broncode, tabel, enz. te verwijzen naar de bron. In de volgende paragraaf een voorbeeld van elk.

%  schreef een van de standaardwerken over sorteer- en zoekalgoritmen. Experten zijn het erover eens dat cloud computing een interessante opportuniteit vormen, zowel voor gebruikers als voor dienstverleners op vlak van informatietechnologie~\autocite{Creeger2009}.

% Let er ook op: het \texttt{cite}-commando voor de punt, dus binnen de zin. Je verwijst meteen naar een bron in de eerste zin die erop gebaseerd is, dus niet pas op het einde van een paragraaf.

