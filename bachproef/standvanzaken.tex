\chapter{\IfLanguageName{dutch}{Stand van zaken}{State of the art}}%
\label{ch:stand-van-zaken}

% Tip: Begin elk hoofdstuk met een paragraaf inleiding die beschrijft hoe
% dit hoofdstuk past binnen het geheel van de bachelorproef. Geef in het
% bijzonder aan wat de link is met het vorige en volgende hoofdstuk.

% Pas na deze inleidende paragraaf komt de eerste sectiehoofding.

De voortdurende vooruitgang in digitale technologieën heeft geleid tot een overvloed aan online hulpmiddelen en toepassingen,
waaronder diverse apps voor het begeleiden van fysieke activiteiten.
Deze sectie, over de huidige stand van zaken, onderzoekt de integratie van online tools voor de ontwikkeling van route-applicaties.

    \section{Route generatie}

    Er is al onderzoek gedaan naar het genereren van routes voor lopers. In deze sectie worden enkele van deze onderzoeken besproken.
    
    \textcite{Loepp2018} hebben onderzoek gedaan naar het genereren van routes voor lopers op basis van verschillende parameters.
    Dit zijn de voorgestelde parameters: Lengte, uniekheid, vorm, verlichting, hoogte, vriendelijkheid, voor voetgangers, bochten, natuur en geschiedenis.
    Zo werd een app, genaamd \emph{Runnerful} een prototypische Android-applicatie, voorgesteld. Er wordt gebruikgemaakt van de OpenStreetMap API om kaartgegevens te verzamelen. 
    Met behulp van randannotaties in deze gegevens (bijvoorbeeld om snelwegen te negeren), 
    wordt vervolgens een grafiek gecreëerd om het route-generatiealgoritme toe te passen.
    Om de kortste routes te vinden tussen via-vertices, wordt een gemodificeerd A* zoekalgoritme gebruikt dat knooppunten 
    die al zijn bezocht bestraft (afstanden worden gevarieerd en verschillende startpunten worden ingesteld). 
    Scores worden berekend voor alle criteria: Voor sommige criteria, zoals Lengte of Vorm, worden scores berekend op basis van de grafiekgegevens zelf. 
    Voor andere, zoals Verlichting of Vriendelijkheid voor voetgangers, wordt vertrouwd op randannotaties verstrekt door OpenStreetMap. 
    Voor Hoogte worden bovendien verzoeken naar de Google Maps Hoogte API gestuurd. Wat betreft de hoeveelheid Natuur, wordt rekening gehouden met omliggende gebieden en hun OpenStreetMap-annotaties:
    Met behulp van een straal-casting algoritme wordt bepaald of segmenten bossen, landbouwgrond of stranden doorkruisen. De verhouding van randen waarvoor dit geldt, 
    bepaalt vervolgens de respectieve score. Bovendien wordt de afstand van elk routepunt tot gebieden die water voorstellen berekend. 
    Als gebruikersinvoer vereist de app in eerste instantie alleen de gewenste routelengte. 
    Vervolgens worden aanbevelingen gegenereerd op basis van de huidige GPS-positie. 
     \autocite{Loepp2018}.

     \hspace{2cm}

    \textcite{Schulze2016} In deze scriptie wordt een model voor route-aantrekkelijkheid geïntroduceerd, 
    samen met verschillende algoritmen om boeiende routes te genereren. 
    Het blijkt dat het vinden van de optimale route alleen praktisch is voor zeer korte afstanden. 
    Voor langere afstanden moet echter snelheid gepriotiseerd worden boven perfectie bij het berekenen van routes. 
    Daarnaast werd een Android-applicatie ontwikkeld die het Driehoeksalgoritme gebruikt om gebruikers routes aan te bieden
    vanuit elk punt in Zwitserland. Deze app kan in principe overal ter wereld worden gebruikt, 
    zolang er voldoende gegevens beschikbaar zijn in OpenStreetMap. Over het algemeen waren de gegevens van goede kwaliteit, 
    met minimale invloed van foutieve gegevens op de gegenereerde routes. 
    In zeldzame gevallen van een slechte route of een pad dat niet geschikt is voor voetgangers of fietsers, 
    kan de gebruiker eenvoudigweg om een nieuwe route vragen.

 
    De app gebruikt OpenStreetMap-gegevens voor routeplanning en weergave. 
    De gebruikersinterface biedt alle nodige tools, inclusief live tracking van gebruikers op de kaart. 
    Gebruikers kunnen een startpunt kiezen, de gewenste lengte selecteren en een route aanvragen, 
    die vervolgens door de server wordt gegenereerd en weergegeven. Hoewel de app momenteel geen routeberekening op het apparaat zelf uitvoert, 
    kan dit wel, zij het langzamer dan op krachtigere computers of servers.

    
    Voor de server wordt Tomcat gebruikt, die twee hoofdfuncties heeft: 
    het hosten van voorbewerkte gegevens en het genereren van routes op aanvraag. 
    Gebruikers sturen een verzoek met de benodigde parameters voor een route, 
    die de server vervolgens genereert met behulp van het Driehoeksalgoritme. 
    Alle gegevens zijn voor Zwitserland voorbewerkt, maar theoretisch kunnen routes in elk gebied ter wereld worden gegenereerd, 
    mits OpenStreetMap voldoende gegevens heeft.
    \autocite{Schulze2016}.

    \hspace{2cm}

    Een ander interessant onderzoek, uitgevoerd door \textcite{Adwinda2020}, 
    heeft geleid tot de ontwikkeling van een Android-applicatie die feedback geeft over de activiteiten van gebruikers tijdens het hardlopen. 
    Deze uitgebreide applicatie biedt een breed scala aan functies, waaronder locatietracking, hartslagberekening, activiteitengrafieken, sociale functies zoals vrienden en feeds, 
    geplande hardlooproutes, pushmeldingen, gamification en sociale media-integratie. Gebruikers kunnen gedetailleerde informatie krijgen over hun hardloopactiviteit, hartslag, 
    hardloopgeschiedenis en prestaties. De app maakt gebruik van de Google Maps API en Firebase voor de database en messaging functies, waardoor een naadloze gebruikerservaring mogelijk is.

    \hspace{2cm}

    \textcite{Gallo2020} 
    In dit paper wordt RunAhead gepresenteerd, 
    een systeem dat head scanning gebruikt als een nieuwe manier om navigatie-informatie op te vragen tijdens het hardlopen. 
    Met RunAhead worden drie verschillende manieren geboden om feedback te geven over de kwaliteit van het bekeken pad: 
    twee op audio gebaseerd (Muziek en Audiocues) en één op haptische (trillingen). 
    RunAhead is getest en vergeleken met een basislijnsysteem dat spraakgestuurde navigatie biedt. 
    De analyse van de verzamelde gegevens toonde aan dat twee versies van RunAhead (Muziek en Haptisch) de voorkeur hadden boven de basislijn. 
    Over het algemeen werd de Haptische versie als minst opdringerig ervaren en kwam als eerste naar voren. 
    In toekomstig werk wordt gepland om RunAhead te verbeteren, rekening houdend met inzichten uit deze studie. 
    Er wordt ook gedacht aan het testen van het systeem in andere contexten en het verkennen van de geschiktheid ervan als verkennend instrument, 
    niet alleen voor turn-by-turn begeleiding.

    \hspace{2cm}

    \textcite{Novack2018} 
    Dit systeem presenteert een methode om aangename wandelroutes te genereren op basis van OpenStreetMap (OSM) data. 
    Gebruikers kunnen interactief de nadruk leggen op factoren zoals groene gebieden, 
    sociale plaatsen en straten met minder verkeerslawaai bij het creëren van een aangepaste route. 
    Het systeem biedt informatie over de lengte, groenheid, aantal sociale plaatsen en totale lengte van lawaaierige straten 
    voor zowel de kortste als aangepaste routes, zodat gebruikers bewustere beslissingen kunnen nemen. 
    Hoewel het systeem momenteel geen rekening houdt met de tijd van de dag bij het berekenen van routes, 
    kan dit in de toekomst worden verbeterd met beschikbare openingstijden van sociale plaatsen. 
    Omdat het systeem volledig gebaseerd is op OSM-data, kan het potentieel wereldwijd worden geïmplementeerd, 
    maar kan het ook afhankelijk zijn van de nauwkeurigheid en beschikbaarheid van OSM-gegevens. 
    Het systeem is nuttig, transparant en intuïtief volgens gebruikersfeedback, 
    waarbij de enquêteresultaten ook richting geven voor verdere ontwikkelingen.
    \autocite{Novack2018}.

    \section{Weather prediction}

    Het weer is een belangrijke factor bij het plannen van een hardlooproute. Het kan de prestaties van de loper beïnvloeden en in extreme gevallen de veiligheid in gevaar brengen.

    Deze tekst door \textcite{dewi2019} illustreert hoe OpenWeatherMap kan gebruikt worden om real-time weerdata te publiceren op Twitter.
    Door platforms zoals Twitter te benutten, wordt directe toegang tot actuele weersinformatie geboden, waardoor het publiek tijdig kan reageren op veranderende weersomstandigheden en potentieel gevaarlijke situaties.
    Met behulp van Python-programmering en gerelateerde tools zoals OAuth, Tweepy en OpenWeatherMap API, wordt het mogelijk om automatisch actuele weergegevens te verzamelen en te verspreiden via sociale media. 
    Dit opent de deur naar efficiënte communicatie van weersomstandigheden, waardoor individuen hun activiteiten kunnen plannen en zich kunnen voorbereiden op weergebonden risico's.

    De conclusie benadrukt de waarde van deze real-time weergegevens in het informeren van mensen en het ondersteunen van weloverwogen beslissingen in het dagelijks leven. 
    Daarnaast wijst het op de noodzaak van verder onderzoek naar geavanceerde algoritmen voor het voorspellen en analyseren van weergegevens, 
    met het oog op mogelijke integratie in mobiele applicaties om weersinformatie nog toegankelijker te maken voor het grote publiek.

    Deze bevindingen belichten de opkomst van technologisch gedreven benaderingen in weersvoorspellingen, waarbij sociale media en programmeertools een essentiële rol spelen 
    in het efficiënt delen van weergegevens en het beschermen van gemeenschappen tegen de risico's van extreme weersomstandigheden.
    \autocite{dewi2019}

    Uit deze tekst blijkt dat het integreren van weersvoorspellingen in een route-genererende applicatie een belangrijke toegevoegde waarde kan bieden voor hardlopers. 
    Verder is OpenWeatherMap een betrouwbare bron van weer-gegevens die kan worden gebruikt om real-time informatie te verstrekken over weersomstandigheden.


    \section{Doelpubliek}

    Welk doelpubliek heeft interesse in een route-genererende applicatie? Welke features zijn belangrijk voor dit doelpubliek?
    Hier wordt onderzocht welke doelgroepen interesse hebben in een dergelijke applicatie.


    De heterogeniteit onder hardlopers maakt het nuttig en aantrekkelijk om hen in groepen te segmenteren om hun AOI's te begrijpen. 
    Segmentatie van consumenten in sport is uitgebreid gedocumenteerd, waarbij studies meestal onderscheid maken tussen consumenten 
    op basis van demografische factoren. Naast demografische kenmerken blijken gedrags- en psychografische variabelen ook 
    verschillende soorten marathonlopers te onderscheiden. Verschillende studies hebben psychografische variabelen, zoals AIO's, 
    gebruikt om hardlopers te clusteren. Verschillende aspecten zoals gezondheid, hardloopidentiteit, persoonlijke doelen, sociale aspecten, 
    verslaving aan hardlopen, commitment, competitie en gemak van de praktijk worden gebruikt om hardlopers te segmenteren.
    Deze studies benadrukken allemaal het belang van AIO's om waardevolle inzichten te krijgen in de behoeften van hardlopers 
    en om typologieën van hardlopers te creëren \textcite{Janssen2020}. 


    In dit onderzoek werd een online vragenlijst, de Eindhoven Running Survey 2016 (ERS2016), 
    gebruikt om gegevens te verzamelen onder deelnemers van het hardloopevenement Eindhoven Marathon. 
    Een totaal van 3727 deelnemers voltooide de vragenlijst volledig (responspercentage van 20,4 \%). 
    Deelnemers waren verdeeld over verschillende afstanden, met de gemiddelde leeftijd van 42,2 jaar. 
    Ongeveer een derde van de deelnemers was vrouw, bijna negen van de tien waren werkzaam en de meerderheid had hoger onderwijs genoten. 
    De sociodemografische kenmerken van de respondenten waren vergelijkbaar met eerdere grootschalige hardloopstudies in West-Europa.\textcite{Janssen2020}.
    Dit betekent dat de resultaten van dit onderzoek waarschijnlijk representatief zijn voor de hardlooppopulatie in West-Europa.


    In dit onderzoek werden 4 types hardlopers geïdentificeerd:
    \begin{itemize}
        \item Recreatieve individuele hardlopers
        \item Sociale competitieve hardlopers
        \item Individuele competitieve hardlopers
        \item Toegewijde hardlopers
    \end{itemize}


    Binnen deze 4 groepen van hardlopers zijn er verschillen in het gebruik van apps en andere apparatuur. 
    De resultaten toonden significante variaties (alleen significante effecten worden beschreven). 
    In relatieve termen waren recreatieve individuele hardlopers de meest enthousiaste app-gebruikers
    en de kleinste groep gebruikers van sporthorloges, terwijl sociale competitieve hardlopers minder app-gebruikers 
    omvatten dan recreatieve individuen, ongeveer hetzelfde als individuele competitieve
    en meer dan toegewijde hardlopers. Sociale competitieve hardlopers omvatten meer gebruikers van sporthorloges 
    dan de recreatieve individuele groep, en minder dan zowel individuele competitieve
    als toegewijde groepen. De laagste proportie niet-gebruikers werd gevonden 
    onder de individuele competitieve hardlopers in vergelijking met de andere types, 
    terwijl zij en de toegewijde hardlopers de hoogste acceptatie van sporthorloges hadden. 
    Ten slotte hadden de toegewijde hardlopers de minste app-gebruikers, 
    en samen met de individuele competitieve hardlopers hadden zij de hoogste proportie gebruikers van sporthorloges.


    De resultaten van dit onderzoek tonen aan dat er verschillende groepen hardlopers zijn die verschillende behoeften hebben. 
    Het is belangrijk om deze verschillen in overweging te nemen bij het ontwikkelen van een route-genererende applicatie.
    Zo kan er bijvoorbeeld een functie toegevoegd worden die het mogelijk maakt om routes te delen met vrienden,
    of een functie die het mogelijk maakt om een trainingsschema aan te maken.

    \section{Design overwegingen}

    In de sectie over Doelpubliek werd reeds besproken dat er verschillende groepen hardlopers zijn die verschillende behoeften hebben.
    Hieronder zitten niet per se enkel verschillen in het gebruik van apps en andere apparatuur, maar ook verschillen in de manier waarop ze gemotiveerd worden.
    Het is belangrijk om deze verschillen in overweging te nemen bij het ontwikkelen van een route-genererende applicatie.


    Hardlopen wordt gekenmerkt door zijn lage drempel en aantrekkelijkheid voor een breed scala aan mensen. 
    Deze heterogeniteit onder hardlopers komt ook tot uiting in de grote aantallen traditionele en thematische hardloopevenementen. 
    Er is echter een hoog uitvalpercentage onder amateurhardlopers, voornamelijk als gevolg van hardloopgerelateerde blessures en verlies van motivatie. 
    Een belangrijke uitdaging blijft het omzetten van intenties in daadwerkelijk langdurig hardloopgedrag. 
    Barrières en drijfveren tussen het maken van de intentie om te gaan hardlopen en het daadwerkelijke hardlopen zelf spelen hierbij een rol. 
    Begrip van hoe amateurhardlopers dagelijkse barrières ervaren en hoe dit hun potentiële runs beïnvloedt, kan waardevol zijn 
    om te identificeren hoe ontwerp- en interactieve technologieën hardlopers beter kunnen ondersteunen voorbij de eigenlijke run.\textcite{Menheere2020}.


    In het onderzoek van \textcite{Menheere2020} werden verschillende ontwerpaanbevelingen gedaan. Hieronder worden enkele van deze aanbevelingen besproken.


    Ontwerpaanbeveling 1: Begeleid zelfspraak en versterk de verwachte beloning van het hardlopen door middel van ontwerp. 
    Illustratief ontwerpconcept: Een interactieve sportmaatje dat dit gesprek begint kan helpen om deze zelfspraak naar een daadwerkelijk gesprek te verplaatsen, 
    anticipatiegevoelens stimuleren en zo bijdragen aan het overtreffen van drijfveren ten opzichte van barrières. 
    Om te meten of het maatje wordt vastgehouden, kan de Hexiwear-prototypingtool worden gebruikt, met een geïntegreerde versnellingsmeter, 
    trilmotor, Bluetooth low energy en een ingebouwde batterij. Een andere strategie zou kunnen bestaan uit het verminderen van de hoeveelheid 
    negatieve zelfspraak door middel van een interactief apparaat dat zelfbewustzijn ondersteunt en die negatieve gedachten omkeert.


    Ontwerpaanbeveling 2: Maak de voorbereidingsrituelen interessanter of plezieriger door middel van ontwerp. 
    Illustratief ontwerpconcept: Een interactieve kledinghanger die de gebruiker overtuigt om hun sportkleding aan te trekken. 
    De gebruiker hangt hun sportkleding aan de hanger, die hardloopintenties detecteert via een verbinding met de agenda van de gebruikers. 
    Gedurende de dag zal de hanger langzaam beginnen te krimpen door de armen van de hanger aan servomotoren te verbinden. 
    Als de gebruiker de kleding uittrekt om te gaan sporten, verschijnt er een motiverend citaat op een geïntegreerd e-inktscherm. 
    Echter, wanneer de sportkleding niet op tijd wordt uitgetrokken, zal de grootte van de hanger een punt bereiken waarop de sportkleding op de grond valt.



    Ontwerpaanbeveling 3: Bied tools aan om hardlopers te helpen hun hardloopsessie van tevoren te visualiseren. 
    Illustratief ontwerpconcept: Een multisensorisch object dat sensaties oproept die verband houden met iemands persoonlijke hardloopervaring. 
    Tussen de runs door zal het object het geluid van je voetstappen en omgeving afspelen, afhankelijk van de vorige hardlooproute. 
    Het object zal ook natuurlijke geuren verspreiden (bijvoorbeeld bomen, modder, gras) en lichtpatronen gebruiken om hardloopbeelden op te roepen.


    Ontwerpaanbeveling 4: Help hardlopers negatieve emoties, zoals frustratie en onzekerheid, die ze ervaren tijdens het hardlopen te overwinnen 
    door middel van ontwerp. 
    Illustratief ontwerpconcept: Een bestaand concept op de markt dat deze ontwerpuitdaging aanpakt, is 'Zombies Run!', 
    een gegamificeerde applicatie die de hardloper onderdompelt in een post-apocalyptische omgeving. 
    Door hun missie en muziek via koptelefoon te horen, moet de speler zombies vermijden en goederen verzamelen om te overleven. 
    Het gamificeren van de hardloopinspanning transformeert negatieve emoties in positieve gevoelens \autocite{Menheere2020}.


    \section{Technologie}


    Verschillende technologieën kunnen worden overwogen voor het ontwikkelen van een route-genererende applicatie. Hieronder worden enkele van deze technologieën besproken:

    \subsection{React Native}

    React Native is een populair open-source framework voor het ontwikkelen van mobiele applicaties. 
    Het stelt ontwikkelaars in staat om native-achtige mobiele apps te bouwen met behulp van JavaScript en React, een bekend JavaScript-bibliotheek voor het bouwen van gebruikersinterfaces. 
    Wat React Native uniek maakt, is dat het een enkele codebase mogelijk maakt voor zowel Android- als iOS-platforms. Hierdoor kan het ontwikkelingsteam efficiënter werken, 
    aangezien ze niet apart hoeven te ontwikkelen voor elk platform. React Native maakt gebruik van native componenten en biedt een snelle ontwikkelingscyclus, 
    waardoor ontwikkelaars snel prototypes kunnen bouwen en itereren. \autocite{react_native_docs}.

    \subsection{Flutter}
    
    Flutter is een ander krachtig open-source framework, ontwikkeld door Google, voor het bouwen van natively gecompileerde applicaties voor mobiele, web- en desktopplatforms. 
    Het maakt gebruik van de programmeertaal Dart en biedt een rijke set van widgets en tools voor het ontwikkelen van mooie en snelle gebruikersinterfaces. 
    Flutter onderscheidt zich door zijn volledig eigen set van widgets, wat resulteert in een consistente look en feel over alle platforms heen. 
    Net als React Native stelt Flutter ontwikkelaars in staat om een enkele codebase te gebruiken voor zowel Android- als iOS-applicaties, wat de ontwikkelingstijd kan verkorten. \autocite{flutter_docs}.

    \subsection{Node.js}
    
    Node.js is een veelgebruikte open-source, cross-platform JavaScript runtime om serverside-applicaties te bouwen. 
    Het wordt vaak gebruikt voor het ontwikkelen van webapplicaties, API's en realtime services vanwege zijn snelle en schaalbare aard. Node.js maakt gebruik van een event-gestuurde, 
    non-blocking I/O-model, waardoor het efficiënt kan omgaan met een groot aantal gelijktijdige verzoeken. Hoewel Node.js niet specifiek is ontworpen voor het bouwen van mobiele applicaties, 
    kan het worden gebruikt als backend-technologie voor het verwerken van verzoeken en het leveren van gegevens aan de frontend van de applicatie. Node.js biedt een breed scala aan modules en frameworks, 
    waardoor ontwikkelaars snel kunnen bouwen en schalen. \autocite{nodejs_docs}.

    \subsection{Expressjs}

    Express is a minimal and flexible Node.js web application framework that provides a robust set of features for web and mobile applications. 
    It is designed to build single-page, multi-page, and hybrid web applications, as well as RESTful APIs. Express is built on top of Node.js and provides a simple and intuitive API for building web servers and applications. \autocite{express_docs}.

    \subsection{MySQL}

    MySQL is een populaire open-source relationele databasebeheersysteem dat wordt gebruikt voor het opslaan en beheren van gestructureerde gegevens.
    Het biedt een krachtige set van functies, waaronder transactiebeheer, gegevensintegriteit, beveiliging en schaalbaarheid. MySQL wordt vaak gebruikt in combinatie met Node.js voor het bouwen van webapplicaties en API's vanwege zijn compatibiliteit en prestaties. 
    \autocite{mysql_docs}.

    \hspace{2cm}

    De keuze voor de backend technologie is Nodejs. Vooral omdat het makkelijk integereerd met de API's en omdat de ontwikkelaar hier al ervaring mee heeft.

    Voor de frontend technologie is de keuze gevallen op React Native. Dit omdat het een populaire technologie is en omdat het een snelle ontwikkelingscyclus biedt. 
    Hier kan de ontwikkelaar ook zijn ervaring in JavaScript en React gebruiken.

    \section{Kosten}

    De kosten voor het ontwikkelen van een route-genererende applicatie kunnen variëren afhankelijk van verschillende factoren, zoals de complexiteit van de applicatie, de functionaliteiten,
    de technologieën die worden gebruikt \ldots Hieronder worden enkele van de kosten besproken die gepaard gaan met het ontwikkelen van deze applicatie.

    \subsection{Overpass API (OpenStreetMap)}

    Deze API biedt gratis toegang tot OpenStreetMap-gegevens, waaronder kaarten, routes en locatiegegevens. Echter omdat dit een gratis service is,
    zijn er beperkingen op het aantal verzoeken dat kan worden gedaan. Voor grotere hoeveelheden verzoeken is het beter om zelf een server te hosten met de OpenStreetMap-gegevens.
    Wat extra kosten met zich meebrengt voor het hosten van de server.

    \subsection{Google Maps API}

    De Google Maps API biedt toegang tot geavanceerde kaart- en locatiediensten, waaronder routebeschrijvingen, geocoding en plaatsen.
    De kosten voor het gebruik van de Google Maps API variëren afhankelijk van het aantal verzoeken en de services die worden gebruikt.
    Google biedt een gratis tegoed van \$200 per maand, maar voor grotere hoeveelheden verzoeken kunnen extra kosten in rekening worden gebracht.
    Voor het genereren van routes is de prijs \$5 of \$10 per 1000 verzoeken, afhankelijk van de gekozen complexiteit van de route.
    Dezelfde prijs van \$5 per 1000 verzoeken geldt voor het ophalen van de hoogtegegevens van de Google Maps API.

    \subsection{OpenWeatherMap API}

    De OpenWeatherMap API biedt toegang tot actuele en historische weergegevens, waaronder temperatuur, neerslag, wind en luchtdruk.
    De kosten voor het gebruik van de OpenWeatherMap API variëren afhankelijk van het aantal verzoeken.
    OpenWeatherMap biedt een gratis abonnement met beperkte toegang tot de API, maar voor grotere hoeveelheden verzoeken kunnen extra kosten in rekening worden gebracht.
    Gratis abonnementen bieden toegang tot 60 verzoeken per minuut, 1 miljoen verzoeken per maand en 5 dagen historische gegevens.
    Het is 0.0014 euro per API call over de limiet. Voor deze applicatie is het genoeg om enkel het huidige weer op te vragen, wat binnen het gratis abonnement zit.


    \subsection{Hosting}

    Voor het hosten van de backend-server en de database zijn er verschillende opties beschikbaar, waaronder cloudproviders zoals AWS, Google Cloud en Firebase.
    Het is ook altijd mogelijk om op een eigen server te hosten, maar dit brengt extra kosten met zich mee voor de hardware en het onderhoud.
    De kosten voor het hosten van een server variëren afhankelijk van de gekozen provider, de serverconfiguratie en het dataverkeer.
    Voor deze applicatie is het aan te raden om te kiezen voor een cloudprovider, omdat dit schaalbaar en kostenefficiënt is.
    Deze cloud-providers bieden verschillende prijsmodellen, waaronder pay-as-you-go en vaste prijzen.
    Op Firebase kost het hosten van een server ongeveer \$0.026/GB voor opslag en \$0.15/GB voor data traffic.
    Bij Google Cloud kost het \$3.00 per miljoen API calls, wanneer er tussen 2 miljoen  en 1 miljard API calls per maand zijn.
    Voor AWS is de prijsstructuur als volgt:
    \begin{table}[h]
        \centering
        \begin{tabular}{lc}
        \toprule
        \textbf{Bereik} & \textbf{Prijs (per miljoen)} \\
        \midrule
        Eerste 333 miljoen & \$3.50 \\
        Volgende 667 miljoen & \$2.80 \\
        Volgende 19 miljard & \$2.38 \\
        Over 20 miljard & \$1.51 \\
        \bottomrule
        \end{tabular}
        \caption{Prijsstructuur}
        \label{tab:prijsstructuur}
    \end{table}





% Dit hoofdstuk bevat je literatuurstudie. De inhoud gaat verder op de inleiding, maar zal het onderwerp van de bachelorproef *diepgaand* uitspitten. De bedoeling is dat de lezer na lezing van dit hoofdstuk helemaal op de hoogte is van de huidige stand van zaken (state-of-the-art) in het onderzoeksdomein. Iemand die niet vertrouwd is met het onderwerp, weet nu voldoende om de rest van het verhaal te kunnen volgen, zonder dat die er nog andere informatie moet over opzoeken \autocite{Pollefliet2011}.

% Je verwijst bij elke bewering die je doet, vakterm die je introduceert, enz.\ naar je bronnen. In \LaTeX{} kan dat met het commando \texttt{$\backslash${textcite\{\}}} of \texttt{$\backslash${autocite\{\}}}. Als argument van het commando geef je de ``sleutel'' van een ``record'' in een bibliografische databank in het Bib\LaTeX{}-formaat (een tekstbestand). Als je expliciet naar de auteur verwijst in de zin (narratieve referentie), gebruik je \texttt{$\backslash${}textcite\{\}}. Soms is de auteursnaam niet expliciet een onderdeel van de zin, dan gebruik je \texttt{$\backslash${}autocite\{\}} (referentie tussen haakjes). Dit gebruik je bv.~bij een citaat, of om in het bijschrift van een overgenomen afbeelding, broncode, tabel, enz. te verwijzen naar de bron. In de volgende paragraaf een voorbeeld van elk.

%  schreef een van de standaardwerken over sorteer- en zoekalgoritmen. Experten zijn het erover eens dat cloud computing een interessante opportuniteit vormen, zowel voor gebruikers als voor dienstverleners op vlak van informatietechnologie~\autocite{Creeger2009}.

% Let er ook op: het \texttt{cite}-commando voor de punt, dus binnen de zin. Je verwijst meteen naar een bron in de eerste zin die erop gebaseerd is, dus niet pas op het einde van een paragraaf.

