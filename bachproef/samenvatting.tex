%%=============================================================================
%% Samenvatting
%%=============================================================================

% TODO: De "abstract" of samenvatting is een kernachtige (~ 1 blz. voor een
% thesis) synthese van het document.
%
% Een goede abstract biedt een kernachtig antwoord op volgende vragen:
%
% 1. Waarover gaat de bachelorproef?
% 2. Waarom heb je er over geschreven?
% 3. Hoe heb je het onderzoek uitgevoerd?
% 4. Wat waren de resultaten? Wat blijkt uit je onderzoek?
% 5. Wat betekenen je resultaten? Wat is de relevantie voor het werkveld?
%
% Daarom bestaat een abstract uit volgende componenten:
%
% - inleiding + kaderen thema
% - probleemstelling
% - (centrale) onderzoeksvraag
% - onderzoeksdoelstelling
% - methodologie
% - resultaten (beperk tot de belangrijkste, relevant voor de onderzoeksvraag)
% - conclusies, aanbevelingen, beperkingen
%
% LET OP! Een samenvatting is GEEN voorwoord!

%%---------- Nederlandse samenvatting -----------------------------------------
%
% TODO: Als je je bachelorproef in het Engels schrijft, moet je eerst een
% Nederlandse samenvatting invoegen. Haal daarvoor onderstaande code uit
% commentaar.
% Wie zijn bachelorproef in het Nederlands schrijft, kan dit negeren, de inhoud
% wordt niet in het document ingevoegd.

\IfLanguageName{english}{%
\selectlanguage{dutch}
\chapter*{Samenvatting}
\lipsum[1-4]
\selectlanguage{english}
}{}

%%---------- Samenvatting -----------------------------------------------------
% De samenvatting in de hoofdtaal van het document

Deze bachelorproef concentreert zich op het onderzoek naar en de ontwikkeling van een kosteloze route-applicatie voor loopfanaten. 
Het hoofddoel is om diverse online tools, met name gratis publieke API's, te verzamelen en te integreren in deze applicatie. 
Dit om looproutes te genereren op basis van verschillende parameters zoals afstand, hoogtemeters, en ondergrond.

Er zijn al verschillende routeapps beschikbaar voor lopers, maar deze bieden niet altijd de mogelijkheid om een route te genereren op basis van verschillende parameters. 
Bestaande route-apps bieden vaak beperkte functionaliteiten of vereisen een betalend abonnement voor toegang tot geavanceerde functies. 
Door een kosteloze alternatieve app te ontwikkelen, wordt voldaan aan de behoeften van loopfanaten, op een gebruiksvriendelijke en gratis manier.

Het onderzoek begon met het verkennen van beschikbare API's die relevant zijn voor route-generatie. Daarnaast zijn populaire route-apps geanalyseerd om inzicht te krijgen in de functies die zij aanbieden. 
Vervolgens is er een proof of concept ontworpen en ontwikkeld, waarbij React Native en een node.js backend werden gebruikt voor implementatie. 
Ten slotte is een kostprijsonderzoek uitgevoerd om de financiële haalbaarheid van de applicatie te evalueren.

De resultaten van het onderzoek omvatten de identificatie van geschikte API's voor route-generatie, de ontwikkeling van een proof of concept, en inzichten uit het kostprijsonderzoek. 
De proof of concept toont aan dat het mogelijk is om een gebruiksvriendelijke en kosteloze route-app te ontwikkelen met behulp van beschikbare technologieën.
De applicatie zal beschikbaar zijn op zowel Android- als iOS-systemen

De resultaten impliceren de haalbaarheid en potentieel van het ontwikkelen van een kosteloze route-app voor loopfanaten. 
Deze applicatie biedt een alternatief voor bestaande betaalde route-apps en opent mogelijkheden voor verdere innovatie. 

\chapter*{\IfLanguageName{dutch}{Samenvatting}{Abstract}}

\lipsum[1-4]
