%---------- Inleiding ---------------------------------------------------------

\section{Introductie}%
\label{sec:introductie}

In de moderne wereld, waarin technologie steeds belangrijker wordt, is het niet meer dan normaal dat er ook technologie bestaat voor sporters. Zo zijn er al verschillende apps beschikbaar voor lopers, zoals Strava, Runkeeper en Nike Run Club.
Deze apps bieden de mogelijkheid om routes te genereren op basis van verschillende parameters, zoals afstand, locatie en moeilijkheidsgraad. Deze functies zijn echter niet altijd toegankelijk via de gratis versie van de app. 
In sommige gevallen moet er een abonnement worden afgesloten om gebruik te kunnen maken van deze functies. Een API, oftewel een interface, maakt communicatie tussen verschillende applicaties mogelijk.
Welke API's zijn er beschikbaar om routes te genereren? Welke combinatie van deze API's is het meest geschikt voor het genereren van routes voor loopfanaten? Welke functies bieden populaire route-apps aan? Welke parameters willen loopfanaten aan een route koppelen? Dit zijn de vragen die in dit onderzoek beantwoord zullen worden.
Dit onderzoek concentreert zich op het verzamelen van diverse online tools, in de vorm van gratis open API's, met als doel ze te integreren in een kosteloze route-app voor loopfanaten. 
Om dit te realiseren zal er een proof of concept ontwikkeld worden die gebruik maakt van de gekozen API's. De applicatie zal ontwikkeld worden in React Native en Node.js. React Native is een framework dat het mogelijk maakt om native apps te ontwikkelen voor Android en iOS\@. Node.js is een JavaScript runtime die het mogelijk maakt om JavaScript code te schrijven buiten de browser.
Deze proof of concept zal beschikbaar zijn op zowel Android- als iOS-systemen.


%---------- Stand van zaken ---------------------------------------------------

\section{State-of-the-art}%
\label{sec:state-of-the-art}

Hier beschrijf je de \emph{state-of-the-art} rondom je gekozen onderzoeksdomein, d.w.z.\ een inleidende, doorlopende tekst over het onderzoeksdomein van je bachelorproef. Je steunt daarbij heel sterk op de professionele \emph{vakliteratuur}, en niet zozeer op populariserende teksten voor een breed publiek. Wat is de huidige stand van zaken in dit domein, en wat zijn nog eventuele open vragen (die misschien de aanleiding waren tot je onderzoeksvraag!)?

Je mag de titel van deze sectie ook aanpassen (literatuurstudie, stand van zaken, enz.). Zijn er al gelijkaardige onderzoeken gevoerd? Wat concluderen ze? Wat is het verschil met jouw onderzoek?

Verwijs bij elke introductie van een term of bewering over het domein naar de vakliteratuur, bijvoorbeeld~\autocite{Hykes2013}! Denk zeker goed na welke werken je refereert en waarom.

Draag zorg voor correcte literatuurverwijzingen! Een bronvermelding hoort thuis \emph{binnen} de zin waar je je op die bron baseert, dus niet er buiten! Maak meteen een verwijzing als je gebruik maakt van een bron. Doe dit dus \emph{niet} aan het einde van een lange paragraaf. Baseer nooit teveel aansluitende tekst op eenzelfde bron.

Als je informatie over bronnen verzamelt in JabRef, zorg er dan voor dat alle nodige info aanwezig is om de bron terug te vinden (zoals uitvoerig besproken in de lessen Research Methods).

% Voor literatuurverwijzingen zijn er twee belangrijke commando's:
% \autocite{KEY} => (Auteur, jaartal) Gebruik dit als de naam van de auteur
%   geen onderdeel is van de zin.
% \textcite{KEY} => Auteur (jaartal)  Gebruik dit als de auteursnaam wel een
%   functie heeft in de zin (bv. ``Uit onderzoek door Doll & Hill (1954) bleek
%   ...'')

Je mag deze sectie nog verder onderverdelen in subsecties als dit de structuur van de tekst kan verduidelijken.

%---------- Methodologie ------------------------------------------------------
\section{Methodologie}%
\label{sec:methodologie}

Eerst wordt een grondige analyse gedaan over welke API's er beschikbaar zijn om routes te genereren. Vervolgens wordt er een vergelijking gemaakt tussen de verschillende API's en wordt er een keuze gemaakt welke API het meest geschikt is voor het genereren van routes. Er zal een combinatie nodig zijn van verschillende API's om de gewenste routes te genereren. 
Er zal een applicatie ontwikkeld worden als \emph{Proof of Concept} die gebruik maakt van de gekozen API's. De applicatie zal ontwikkeld worden in React Native en Node.js. React Native is een framework dat het mogelijk maakt om native apps te ontwikkelen voor Android en iOS\@. Node.js is een JavaScript runtime die het mogelijk maakt om JavaScript code te schrijven buiten de browser.

Hier beschrijf je hoe je van plan bent het onderzoek te voeren. Welke onderzoekstechniek ga je toepassen om elk van je onderzoeksvragen te beantwoorden? Gebruik je hiervoor literatuurstudie, interviews met belanghebbenden (bv.~voor requirements-analyse), experimenten, simulaties, vergelijkende studie, risico-analyse, PoC, \ldots?

Valt je onderwerp onder één van de typische soorten bachelorproeven die besproken zijn in de lessen Research Methods (bv.\ vergelijkende studie of risico-analyse)? Zorg er dan ook voor dat we duidelijk de verschillende stappen terug vinden die we verwachten in dit soort onderzoek!

Vermijd onderzoekstechnieken die geen objectieve, meetbare resultaten kunnen opleveren. Enquêtes, bijvoorbeeld, zijn voor een bachelorproef informatica meestal \textbf{niet geschikt}. De antwoorden zijn eerder meningen dan feiten en in de praktijk blijkt het ook bijzonder moeilijk om voldoende respondenten te vinden. Studenten die een enquête willen voeren, hebben meestal ook geen goede definitie van de populatie, waardoor ook niet kan aangetoond worden dat eventuele resultaten representatief zijn.

Uit dit onderdeel moet duidelijk naar voor komen dat je bachelorproef ook technisch voldoen\-de diepgang zal bevatten. Het zou niet kloppen als een bachelorproef informatica ook door bv.\ een student marketing zou kunnen uitgevoerd worden.

Je beschrijft ook al welke tools (hardware, software, diensten, \ldots) je denkt hiervoor te gebruiken of te ontwikkelen.

Probeer ook een tijdschatting te maken. Hoe lang zal je met elke fase van je onderzoek bezig zijn en wat zijn de concrete \emph{deliverables} in elke fase?

%---------- Verwachte resultaten ----------------------------------------------
\section{Verwacht resultaat, conclusie}%
\label{sec:verwachte_resultaten}

Hier komen mockups van de applicatie, een beschrijving van de functionaliteiten en een kostenanalyse.

